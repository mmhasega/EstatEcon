% Options for packages loaded elsewhere
\PassOptionsToPackage{unicode}{hyperref}
\PassOptionsToPackage{hyphens}{url}
%
\documentclass[
]{book}
\usepackage{lmodern}
\usepackage{amssymb,amsmath}
\usepackage{ifxetex,ifluatex}
\ifnum 0\ifxetex 1\fi\ifluatex 1\fi=0 % if pdftex
  \usepackage[T1]{fontenc}
  \usepackage[utf8]{inputenc}
  \usepackage{textcomp} % provide euro and other symbols
\else % if luatex or xetex
  \usepackage{unicode-math}
  \defaultfontfeatures{Scale=MatchLowercase}
  \defaultfontfeatures[\rmfamily]{Ligatures=TeX,Scale=1}
\fi
% Use upquote if available, for straight quotes in verbatim environments
\IfFileExists{upquote.sty}{\usepackage{upquote}}{}
\IfFileExists{microtype.sty}{% use microtype if available
  \usepackage[]{microtype}
  \UseMicrotypeSet[protrusion]{basicmath} % disable protrusion for tt fonts
}{}
\makeatletter
\@ifundefined{KOMAClassName}{% if non-KOMA class
  \IfFileExists{parskip.sty}{%
    \usepackage{parskip}
  }{% else
    \setlength{\parindent}{0pt}
    \setlength{\parskip}{6pt plus 2pt minus 1pt}}
}{% if KOMA class
  \KOMAoptions{parskip=half}}
\makeatother
\usepackage{xcolor}
\IfFileExists{xurl.sty}{\usepackage{xurl}}{} % add URL line breaks if available
\IfFileExists{bookmark.sty}{\usepackage{bookmark}}{\usepackage{hyperref}}
\hypersetup{
  pdftitle={Notas de aulas de Estatística Econômica},
  pdfauthor={Marcos Minoru Hasegawa},
  hidelinks,
  pdfcreator={LaTeX via pandoc}}
\urlstyle{same} % disable monospaced font for URLs
\usepackage{color}
\usepackage{fancyvrb}
\newcommand{\VerbBar}{|}
\newcommand{\VERB}{\Verb[commandchars=\\\{\}]}
\DefineVerbatimEnvironment{Highlighting}{Verbatim}{commandchars=\\\{\}}
% Add ',fontsize=\small' for more characters per line
\usepackage{framed}
\definecolor{shadecolor}{RGB}{248,248,248}
\newenvironment{Shaded}{\begin{snugshade}}{\end{snugshade}}
\newcommand{\AlertTok}[1]{\textcolor[rgb]{0.94,0.16,0.16}{#1}}
\newcommand{\AnnotationTok}[1]{\textcolor[rgb]{0.56,0.35,0.01}{\textbf{\textit{#1}}}}
\newcommand{\AttributeTok}[1]{\textcolor[rgb]{0.77,0.63,0.00}{#1}}
\newcommand{\BaseNTok}[1]{\textcolor[rgb]{0.00,0.00,0.81}{#1}}
\newcommand{\BuiltInTok}[1]{#1}
\newcommand{\CharTok}[1]{\textcolor[rgb]{0.31,0.60,0.02}{#1}}
\newcommand{\CommentTok}[1]{\textcolor[rgb]{0.56,0.35,0.01}{\textit{#1}}}
\newcommand{\CommentVarTok}[1]{\textcolor[rgb]{0.56,0.35,0.01}{\textbf{\textit{#1}}}}
\newcommand{\ConstantTok}[1]{\textcolor[rgb]{0.00,0.00,0.00}{#1}}
\newcommand{\ControlFlowTok}[1]{\textcolor[rgb]{0.13,0.29,0.53}{\textbf{#1}}}
\newcommand{\DataTypeTok}[1]{\textcolor[rgb]{0.13,0.29,0.53}{#1}}
\newcommand{\DecValTok}[1]{\textcolor[rgb]{0.00,0.00,0.81}{#1}}
\newcommand{\DocumentationTok}[1]{\textcolor[rgb]{0.56,0.35,0.01}{\textbf{\textit{#1}}}}
\newcommand{\ErrorTok}[1]{\textcolor[rgb]{0.64,0.00,0.00}{\textbf{#1}}}
\newcommand{\ExtensionTok}[1]{#1}
\newcommand{\FloatTok}[1]{\textcolor[rgb]{0.00,0.00,0.81}{#1}}
\newcommand{\FunctionTok}[1]{\textcolor[rgb]{0.00,0.00,0.00}{#1}}
\newcommand{\ImportTok}[1]{#1}
\newcommand{\InformationTok}[1]{\textcolor[rgb]{0.56,0.35,0.01}{\textbf{\textit{#1}}}}
\newcommand{\KeywordTok}[1]{\textcolor[rgb]{0.13,0.29,0.53}{\textbf{#1}}}
\newcommand{\NormalTok}[1]{#1}
\newcommand{\OperatorTok}[1]{\textcolor[rgb]{0.81,0.36,0.00}{\textbf{#1}}}
\newcommand{\OtherTok}[1]{\textcolor[rgb]{0.56,0.35,0.01}{#1}}
\newcommand{\PreprocessorTok}[1]{\textcolor[rgb]{0.56,0.35,0.01}{\textit{#1}}}
\newcommand{\RegionMarkerTok}[1]{#1}
\newcommand{\SpecialCharTok}[1]{\textcolor[rgb]{0.00,0.00,0.00}{#1}}
\newcommand{\SpecialStringTok}[1]{\textcolor[rgb]{0.31,0.60,0.02}{#1}}
\newcommand{\StringTok}[1]{\textcolor[rgb]{0.31,0.60,0.02}{#1}}
\newcommand{\VariableTok}[1]{\textcolor[rgb]{0.00,0.00,0.00}{#1}}
\newcommand{\VerbatimStringTok}[1]{\textcolor[rgb]{0.31,0.60,0.02}{#1}}
\newcommand{\WarningTok}[1]{\textcolor[rgb]{0.56,0.35,0.01}{\textbf{\textit{#1}}}}
\usepackage{longtable,booktabs}
% Correct order of tables after \paragraph or \subparagraph
\usepackage{etoolbox}
\makeatletter
\patchcmd\longtable{\par}{\if@noskipsec\mbox{}\fi\par}{}{}
\makeatother
% Allow footnotes in longtable head/foot
\IfFileExists{footnotehyper.sty}{\usepackage{footnotehyper}}{\usepackage{footnote}}
\makesavenoteenv{longtable}
\usepackage{graphicx,grffile}
\makeatletter
\def\maxwidth{\ifdim\Gin@nat@width>\linewidth\linewidth\else\Gin@nat@width\fi}
\def\maxheight{\ifdim\Gin@nat@height>\textheight\textheight\else\Gin@nat@height\fi}
\makeatother
% Scale images if necessary, so that they will not overflow the page
% margins by default, and it is still possible to overwrite the defaults
% using explicit options in \includegraphics[width, height, ...]{}
\setkeys{Gin}{width=\maxwidth,height=\maxheight,keepaspectratio}
% Set default figure placement to htbp
\makeatletter
\def\fps@figure{htbp}
\makeatother
\setlength{\emergencystretch}{3em} % prevent overfull lines
\providecommand{\tightlist}{%
  \setlength{\itemsep}{0pt}\setlength{\parskip}{0pt}}
\setcounter{secnumdepth}{5}
\usepackage[brazil]{babel}
\usepackage[utf8]{inputenc}
\usepackage[T1]{fontenc}
\usepackage{lmodern}
\usepackage{amsmath,amssymb,amsthm,adjustbox,mathtools,bm,cool}
\usepackage{natbib}
\usepackage{graphics,graphicx,import,color,float}
\usepackage{url,booktabs,siunitx}
\usepackage[]{natbib}
\bibliographystyle{abnt}

\title{Notas de aulas de Estatística Econômica}
\author{Marcos Minoru Hasegawa}
\date{2020-08-31}

\begin{document}
\maketitle

{
\setcounter{tocdepth}{1}
\tableofcontents
}
\hypertarget{licenuxe7a}{%
\chapter*{Licença}\label{licenuxe7a}}
\addcontentsline{toc}{chapter}{Licença}

Como está descrito no repositório, os poucos códigos originais desenvolvidos ao longo do texto estão sob a licença \textbf{GNU GPLv3} .

O texto e as artes gráficas elaboradas de forma original estão sob licença \textbf{Creative Commons BY-NC-SA 4.0}.

\hypertarget{sobre-o-material}{%
\chapter*{Sobre o material}\label{sobre-o-material}}
\addcontentsline{toc}{chapter}{Sobre o material}

A situação especial causada pela pandemia da COVID-19 forçou a muitos professores criarem materiais para facilitar aulas remotas das suas disciplinas. A disciplina SE305 Estatística Econômica e Introdução à Econometria da UFPR não poderia ser diferente. Então, o objetivo deste material é de suprir a falta das bibliografias básicas na sua versão digital com a disponibilização de forma digital e gratuita o que seria o material das notas das aulas da disciplina de Estatística Econômica. Não é o ideal, mas a ideia é melhorar o material com tempo.

\hypertarget{sobre-o-autor}{%
\chapter*{Sobre o Autor}\label{sobre-o-autor}}
\addcontentsline{toc}{chapter}{Sobre o Autor}

Professor do Departamento de Economia da Universidade Federal do Paraná. Engenheiro Agrônomo pela UNESP/Jaboticabal, Mestrado em Economia Agrária pela ESALQ/USP e Doutorado em Economia Aplicada pela ESALQ/USP, é um dos professores responsáveis pelas disciplinas de SE305 Estatística Econômica e Introdução à Econometria e SE308 Econometria ambas do curso de Economia da Universidade Federal do Paraná (UFPR).

\hypertarget{medidas-de-posiuxe7uxe3o-e-dispersuxe3o}{%
\chapter{Medidas de posição e dispersão}\label{medidas-de-posiuxe7uxe3o-e-dispersuxe3o}}

Este tópico está baseado no material de \citet{Sartoris2013}.

\hypertarget{variuxe1vel-aleatuxf3ria}{%
\section{Variável Aleatória}\label{variuxe1vel-aleatuxf3ria}}

\begin{itemize}
\tightlist
\item
  variável aleatória (v.a.) é uma variável que está associada a uma \emph{distribuição de probabilidade}.
\item
  O resultado do lançamento de uma dado, que poder ser qualquer número de 1 a 6, está associada a uma probabilidade de \(1/6\).
\end{itemize}

\hypertarget{muxe9dia-aritmuxe9tica}{%
\section{Média aritmética}\label{muxe9dia-aritmuxe9tica}}

\begin{equation}
    \overline{X} = \frac{1}{n} \sum_{i=1}^{n} X_i
    \label{eq:eq11}
\end{equation}
onde \(i =1, ...,n\)

\hypertarget{exemplo-1}{%
\subsection{Exemplo 1}\label{exemplo-1}}

Qual é a média aritmética de um grupo de cinco pessoas cujas idades são em ordem crescente, 21,23,25,28 e 31. Para responder, basta aplicar \eqref{eq:eq11}.

\begin{equation*}
  \overline{X} = \frac{21+23+25+28+31}{5} = 25,6
\end{equation*}

\hypertarget{exemplo-1-no-r}{%
\subsection{EXemplo 1 no R}\label{exemplo-1-no-r}}

\begin{Shaded}
\begin{Highlighting}[]
\NormalTok{X <-}\StringTok{ }\KeywordTok{c}\NormalTok{(}\DecValTok{21}\NormalTok{, }\DecValTok{23}\NormalTok{, }\DecValTok{25}\NormalTok{, }\DecValTok{28}\NormalTok{, }\DecValTok{31}\NormalTok{)}
\NormalTok{X}
\end{Highlighting}
\end{Shaded}

\begin{verbatim}
## [1] 21 23 25 28 31
\end{verbatim}

\begin{Shaded}
\begin{Highlighting}[]
\NormalTok{mediaX <-}\StringTok{ }\KeywordTok{mean}\NormalTok{(X)}
\NormalTok{mediaX}
\end{Highlighting}
\end{Shaded}

\begin{verbatim}
## [1] 25,6
\end{verbatim}

\hypertarget{exemplo-2}{%
\subsection{Exemplo 2}\label{exemplo-2}}

Qual é a média aritmética de três provas realizadas por um aluno, cujas notas
foram 4,6 e 8. Para responder, basta aplicar \eqref{eq:eq11}.

\begin{equation*}
  \overline{X} = \frac{4+6+8}{3} = 6
\end{equation*}

\hypertarget{exemplo-2-no-r}{%
\subsection{Exemplo 2 no R}\label{exemplo-2-no-r}}

\begin{Shaded}
\begin{Highlighting}[]
\NormalTok{X2 <-}\StringTok{ }\KeywordTok{c}\NormalTok{(}\DecValTok{4}\NormalTok{, }\DecValTok{6}\NormalTok{, }\DecValTok{8}\NormalTok{)}
\NormalTok{X2}
\end{Highlighting}
\end{Shaded}

\begin{verbatim}
## [1] 4 6 8
\end{verbatim}

\begin{Shaded}
\begin{Highlighting}[]
\NormalTok{mediaX2 <-}\StringTok{ }\KeywordTok{mean}\NormalTok{(X2)}
\NormalTok{mediaX2}
\end{Highlighting}
\end{Shaded}

\begin{verbatim}
## [1] 6
\end{verbatim}

\hypertarget{muxe9dia-ponderada}{%
\section{Média Ponderada}\label{muxe9dia-ponderada}}

\begin{equation}
    \overline{X} = \frac{1}{\sum_{i=1}^{n}w_i} \sum_{i=1}^{n} w_i X_i
    (#eq:eq12)
\end{equation}
onde \(w_i\) é a ponderação ou peso associado a iésimo valor de \(X\).

Podemos escrever na forma de frequência relativa dos valores da variável \(X\):

\begin{equation}
  f_i = \frac{w_i}{\sum_{i=1}^{n}w_i}
  (#eq:eq13)
\end{equation}

\hypertarget{exemplo-3}{%
\subsection{Exemplo 3}\label{exemplo-3}}

Qual é a média aritmética de um grupo de vinte alunos, oito com 22 anos, sete
de 23 anos, três de 25 anos, um de 28 anos e um de 30 anos. Para responder,
basta aplicar \eqref{eq:eq12}.

\begin{equation*}
  \overline{X} = \frac{22\times 8 + 23\times 7 + 25 \times 3 + 28 \times 1 + 
  30 \times 1}{20} = 23,5
\end{equation*}

\hypertarget{exemplo-3-no-r}{%
\subsection{Exemplo 3 no R}\label{exemplo-3-no-r}}

\begin{Shaded}
\begin{Highlighting}[]
\NormalTok{X3 <-}\StringTok{ }\KeywordTok{c}\NormalTok{(}\DecValTok{22}\NormalTok{, }\DecValTok{23}\NormalTok{, }\DecValTok{25}\NormalTok{, }\DecValTok{28}\NormalTok{, }\DecValTok{30}\NormalTok{)}
\NormalTok{X3}
\end{Highlighting}
\end{Shaded}

\begin{verbatim}
## [1] 22 23 25 28 30
\end{verbatim}

\begin{Shaded}
\begin{Highlighting}[]
\NormalTok{w3 <-}\StringTok{ }\KeywordTok{c}\NormalTok{(}\DecValTok{8}\NormalTok{, }\DecValTok{7}\NormalTok{, }\DecValTok{3}\NormalTok{, }\DecValTok{1}\NormalTok{, }\DecValTok{1}\NormalTok{)}
\NormalTok{w3}
\end{Highlighting}
\end{Shaded}

\begin{verbatim}
## [1] 8 7 3 1 1
\end{verbatim}

\begin{Shaded}
\begin{Highlighting}[]
\NormalTok{wX3 <-}\StringTok{ }\NormalTok{w3 }\OperatorTok{*}\StringTok{ }\NormalTok{X3}
\NormalTok{mediaX3 <-}\StringTok{ }\KeywordTok{sum}\NormalTok{(wX3)}\OperatorTok{/}\KeywordTok{sum}\NormalTok{(w3)}
\NormalTok{mediaX3}
\end{Highlighting}
\end{Shaded}

\begin{verbatim}
## [1] 23,5
\end{verbatim}

\hypertarget{exemplo-4}{%
\subsection{Exemplo 4}\label{exemplo-4}}

Qual é a média ponderada de três provas realizadas por um aluno, cujas notas foram 4, 6 e 8. A primeira prova tem peso igual a 1, a segunda tem peso igual a 2 e a terceira tem peso igual a 3. Para responder, basta aplicar \eqref{eq:eq12}.

\begin{equation*}
  \overline{X} = \frac{4 \times 1 + 6 \times 2 + 8 \times 3}{1 + 2 + 3} \cong 6,7
\end{equation*}

\hypertarget{exemplo-4-no-r}{%
\subsection{Exemplo 4 no R}\label{exemplo-4-no-r}}

\begin{Shaded}
\begin{Highlighting}[]
\NormalTok{X4 <-}\StringTok{ }\KeywordTok{c}\NormalTok{(}\DecValTok{4}\NormalTok{, }\DecValTok{6}\NormalTok{, }\DecValTok{8}\NormalTok{)}
\NormalTok{X4}
\end{Highlighting}
\end{Shaded}

\begin{verbatim}
## [1] 4 6 8
\end{verbatim}

\begin{Shaded}
\begin{Highlighting}[]
\NormalTok{w4 <-}\StringTok{ }\KeywordTok{c}\NormalTok{(}\DecValTok{1}\NormalTok{, }\DecValTok{2}\NormalTok{, }\DecValTok{3}\NormalTok{)}
\NormalTok{w4}
\end{Highlighting}
\end{Shaded}

\begin{verbatim}
## [1] 1 2 3
\end{verbatim}

\begin{Shaded}
\begin{Highlighting}[]
\NormalTok{wX4 <-}\StringTok{ }\NormalTok{w4 }\OperatorTok{*}\StringTok{ }\NormalTok{X4}
\NormalTok{mediaX4 <-}\StringTok{ }\KeywordTok{sum}\NormalTok{(wX4)}\OperatorTok{/}\KeywordTok{sum}\NormalTok{(w4)}
\KeywordTok{round}\NormalTok{(mediaX4, }\DataTypeTok{digits =} \DecValTok{1}\NormalTok{)}
\end{Highlighting}
\end{Shaded}

\begin{verbatim}
## [1] 6,7
\end{verbatim}

\hypertarget{revisuxe3o-de-literatura}{%
\chapter{Revisão de Literatura}\label{revisuxe3o-de-literatura}}

Aqui o estado da arte mundo afora.

\hypertarget{metodologia}{%
\chapter{Metodologia}\label{metodologia}}

We describe our methods in this chapter.

\hypertarget{aplicauxe7uxf5es}{%
\chapter{Aplicações}\label{aplicauxe7uxf5es}}

Some \emph{significant} applications are demonstrated in this chapter.

\hypertarget{exemplo-um}{%
\section{Exemplo um}\label{exemplo-um}}

\hypertarget{examplo-dois}{%
\section{Examplo dois}\label{examplo-dois}}

\hypertarget{considerauxe7uxf5es-finais}{%
\chapter{Considerações Finais}\label{considerauxe7uxf5es-finais}}

Terminado um excelente livro digital.

  \bibliography{estatecon.bib,book.bib,packages.bib}

\end{document}
