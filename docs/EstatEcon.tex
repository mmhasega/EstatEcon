% Options for packages loaded elsewhere
\PassOptionsToPackage{unicode}{hyperref}
\PassOptionsToPackage{hyphens}{url}
%
\documentclass[
]{book}
\usepackage{lmodern}
\usepackage{amssymb,amsmath}
\usepackage{ifxetex,ifluatex}
\ifnum 0\ifxetex 1\fi\ifluatex 1\fi=0 % if pdftex
  \usepackage[T1]{fontenc}
  \usepackage[utf8]{inputenc}
  \usepackage{textcomp} % provide euro and other symbols
\else % if luatex or xetex
  \usepackage{unicode-math}
  \defaultfontfeatures{Scale=MatchLowercase}
  \defaultfontfeatures[\rmfamily]{Ligatures=TeX,Scale=1}
\fi
% Use upquote if available, for straight quotes in verbatim environments
\IfFileExists{upquote.sty}{\usepackage{upquote}}{}
\IfFileExists{microtype.sty}{% use microtype if available
  \usepackage[]{microtype}
  \UseMicrotypeSet[protrusion]{basicmath} % disable protrusion for tt fonts
}{}
\makeatletter
\@ifundefined{KOMAClassName}{% if non-KOMA class
  \IfFileExists{parskip.sty}{%
    \usepackage{parskip}
  }{% else
    \setlength{\parindent}{0pt}
    \setlength{\parskip}{6pt plus 2pt minus 1pt}}
}{% if KOMA class
  \KOMAoptions{parskip=half}}
\makeatother
\usepackage{xcolor}
\IfFileExists{xurl.sty}{\usepackage{xurl}}{} % add URL line breaks if available
\IfFileExists{bookmark.sty}{\usepackage{bookmark}}{\usepackage{hyperref}}
\hypersetup{
  pdftitle={Notas de aulas de Estatística Econômica},
  pdfauthor={Marcos Minoru Hasegawa},
  hidelinks,
  pdfcreator={LaTeX via pandoc}}
\urlstyle{same} % disable monospaced font for URLs
\usepackage{color}
\usepackage{fancyvrb}
\newcommand{\VerbBar}{|}
\newcommand{\VERB}{\Verb[commandchars=\\\{\}]}
\DefineVerbatimEnvironment{Highlighting}{Verbatim}{commandchars=\\\{\}}
% Add ',fontsize=\small' for more characters per line
\usepackage{framed}
\definecolor{shadecolor}{RGB}{248,248,248}
\newenvironment{Shaded}{\begin{snugshade}}{\end{snugshade}}
\newcommand{\AlertTok}[1]{\textcolor[rgb]{0.94,0.16,0.16}{#1}}
\newcommand{\AnnotationTok}[1]{\textcolor[rgb]{0.56,0.35,0.01}{\textbf{\textit{#1}}}}
\newcommand{\AttributeTok}[1]{\textcolor[rgb]{0.77,0.63,0.00}{#1}}
\newcommand{\BaseNTok}[1]{\textcolor[rgb]{0.00,0.00,0.81}{#1}}
\newcommand{\BuiltInTok}[1]{#1}
\newcommand{\CharTok}[1]{\textcolor[rgb]{0.31,0.60,0.02}{#1}}
\newcommand{\CommentTok}[1]{\textcolor[rgb]{0.56,0.35,0.01}{\textit{#1}}}
\newcommand{\CommentVarTok}[1]{\textcolor[rgb]{0.56,0.35,0.01}{\textbf{\textit{#1}}}}
\newcommand{\ConstantTok}[1]{\textcolor[rgb]{0.00,0.00,0.00}{#1}}
\newcommand{\ControlFlowTok}[1]{\textcolor[rgb]{0.13,0.29,0.53}{\textbf{#1}}}
\newcommand{\DataTypeTok}[1]{\textcolor[rgb]{0.13,0.29,0.53}{#1}}
\newcommand{\DecValTok}[1]{\textcolor[rgb]{0.00,0.00,0.81}{#1}}
\newcommand{\DocumentationTok}[1]{\textcolor[rgb]{0.56,0.35,0.01}{\textbf{\textit{#1}}}}
\newcommand{\ErrorTok}[1]{\textcolor[rgb]{0.64,0.00,0.00}{\textbf{#1}}}
\newcommand{\ExtensionTok}[1]{#1}
\newcommand{\FloatTok}[1]{\textcolor[rgb]{0.00,0.00,0.81}{#1}}
\newcommand{\FunctionTok}[1]{\textcolor[rgb]{0.00,0.00,0.00}{#1}}
\newcommand{\ImportTok}[1]{#1}
\newcommand{\InformationTok}[1]{\textcolor[rgb]{0.56,0.35,0.01}{\textbf{\textit{#1}}}}
\newcommand{\KeywordTok}[1]{\textcolor[rgb]{0.13,0.29,0.53}{\textbf{#1}}}
\newcommand{\NormalTok}[1]{#1}
\newcommand{\OperatorTok}[1]{\textcolor[rgb]{0.81,0.36,0.00}{\textbf{#1}}}
\newcommand{\OtherTok}[1]{\textcolor[rgb]{0.56,0.35,0.01}{#1}}
\newcommand{\PreprocessorTok}[1]{\textcolor[rgb]{0.56,0.35,0.01}{\textit{#1}}}
\newcommand{\RegionMarkerTok}[1]{#1}
\newcommand{\SpecialCharTok}[1]{\textcolor[rgb]{0.00,0.00,0.00}{#1}}
\newcommand{\SpecialStringTok}[1]{\textcolor[rgb]{0.31,0.60,0.02}{#1}}
\newcommand{\StringTok}[1]{\textcolor[rgb]{0.31,0.60,0.02}{#1}}
\newcommand{\VariableTok}[1]{\textcolor[rgb]{0.00,0.00,0.00}{#1}}
\newcommand{\VerbatimStringTok}[1]{\textcolor[rgb]{0.31,0.60,0.02}{#1}}
\newcommand{\WarningTok}[1]{\textcolor[rgb]{0.56,0.35,0.01}{\textbf{\textit{#1}}}}
\usepackage{longtable,booktabs}
% Correct order of tables after \paragraph or \subparagraph
\usepackage{etoolbox}
\makeatletter
\patchcmd\longtable{\par}{\if@noskipsec\mbox{}\fi\par}{}{}
\makeatother
% Allow footnotes in longtable head/foot
\IfFileExists{footnotehyper.sty}{\usepackage{footnotehyper}}{\usepackage{footnote}}
\makesavenoteenv{longtable}
\usepackage{graphicx,grffile}
\makeatletter
\def\maxwidth{\ifdim\Gin@nat@width>\linewidth\linewidth\else\Gin@nat@width\fi}
\def\maxheight{\ifdim\Gin@nat@height>\textheight\textheight\else\Gin@nat@height\fi}
\makeatother
% Scale images if necessary, so that they will not overflow the page
% margins by default, and it is still possible to overwrite the defaults
% using explicit options in \includegraphics[width, height, ...]{}
\setkeys{Gin}{width=\maxwidth,height=\maxheight,keepaspectratio}
% Set default figure placement to htbp
\makeatletter
\def\fps@figure{htbp}
\makeatother
\setlength{\emergencystretch}{3em} % prevent overfull lines
\providecommand{\tightlist}{%
  \setlength{\itemsep}{0pt}\setlength{\parskip}{0pt}}
\setcounter{secnumdepth}{5}
\usepackage[brazil]{babel}
\usepackage[utf8]{inputenc}
\usepackage[T1]{fontenc}
\usepackage{lmodern}
\usepackage{amsmath,amssymb,amsthm,adjustbox,mathtools,bm,cool}
\usepackage{natbib}
\usepackage{graphics,graphicx,import,color,float}
\usepackage{url,booktabs,siunitx}
\usepackage[]{natbib}
\bibliographystyle{apalike}

\title{Notas de aulas de Estatística Econômica}
\author{Marcos Minoru Hasegawa}
\date{2020-09-08}

\begin{document}
\maketitle

{
\setcounter{tocdepth}{1}
\tableofcontents
}
\hypertarget{licenuxe7a}{%
\chapter*{Licença}\label{licenuxe7a}}
\addcontentsline{toc}{chapter}{Licença}

Como está descrito no repositório, os poucos códigos originais desenvolvidos ao longo do texto estão sob a licença \textbf{GNU GPLv3} .

O texto e as artes gráficas elaboradas de forma original estão sob licença \textbf{Creative Commons BY-NC-SA 4.0}.

\hypertarget{sobre-o-material}{%
\chapter*{Sobre o material}\label{sobre-o-material}}
\addcontentsline{toc}{chapter}{Sobre o material}

A situação especial causada pela pandemia da COVID-19 forçou a muitos professores criarem materiais para facilitar aulas remotas das suas disciplinas. A disciplina SE305 Estatística Econômica e Introdução à Econometria da UFPR não poderia ser diferente. Então, o objetivo deste material é de suprir a falta das bibliografias básicas na sua versão digital com a disponibilização de forma digital e gratuita o que seria o material das notas das aulas da disciplina de Estatística Econômica. Não é o ideal, mas a ideia é melhorar o material com tempo.

\hypertarget{sobre-o-autor}{%
\chapter*{Sobre o Autor}\label{sobre-o-autor}}
\addcontentsline{toc}{chapter}{Sobre o Autor}

Professor do Departamento de Economia da Universidade Federal do Paraná. Engenheiro Agrônomo pela UNESP/Jaboticabal, Mestrado em Economia Agrária pela ESALQ/USP e Doutorado em Economia Aplicada pela ESALQ/USP, é um dos professores responsáveis pelas disciplinas de SE305 Estatística Econômica e Introdução à Econometria e SE308 Econometria ambas do curso de Economia da Universidade Federal do Paraná (UFPR).

\hypertarget{medidas-de-posiuxe7uxe3o-e-dispersuxe3o}{%
\chapter{Medidas de posição e dispersão}\label{medidas-de-posiuxe7uxe3o-e-dispersuxe3o}}

Este tópico está baseado no material de \citet{Sartoris2013}.

\hypertarget{variuxe1vel-aleatuxf3ria}{%
\section{Variável Aleatória}\label{variuxe1vel-aleatuxf3ria}}

\begin{itemize}
\tightlist
\item
  variável aleatória (v.a.) é uma variável que está associada a uma \emph{distribuição de probabilidade}.
\item
  Ou seja, cada valor da v.a. está associada a uma probabilidade.
\item
  O resultado do lançamento de uma dado, que poder ser qualquer número de 1 a 6, está associada a uma probabilidade de \(1/6\).
\end{itemize}

\hypertarget{muxe9dia-aritmuxe9tica-simples}{%
\section{Média Aritmética Simples}\label{muxe9dia-aritmuxe9tica-simples}}

\begin{equation}
    \overline{X} = \frac{1}{n} \sum_{i=1}^{n} X_i
    \label{eq:mediaaritmeticasimples}
\end{equation}
onde \(i =1, ...,n\)

\hypertarget{exemplo-1}{%
\subsection{Exemplo 1}\label{exemplo-1}}

Qual é a média aritmética de um grupo de cinco pessoas cujas idades são em ordem crescente, 21,23,25,28 e 31. Para responder, basta aplicar \eqref{eq:mediaaritmeticasimples}.

\begin{equation*}
  \overline{X} = \frac{21+23+25+28+31}{5} = 25,6
\end{equation*}

\hypertarget{exemplo-1-no-r}{%
\subsection{EXemplo 1 no R}\label{exemplo-1-no-r}}

\begin{Shaded}
\begin{Highlighting}[]
\NormalTok{X <-}\StringTok{ }\KeywordTok{c}\NormalTok{(}\DecValTok{21}\NormalTok{, }\DecValTok{23}\NormalTok{, }\DecValTok{25}\NormalTok{, }\DecValTok{28}\NormalTok{, }\DecValTok{31}\NormalTok{)}
\NormalTok{X}
\end{Highlighting}
\end{Shaded}

\begin{verbatim}
## [1] 21 23 25 28 31
\end{verbatim}

\begin{Shaded}
\begin{Highlighting}[]
\NormalTok{mediaX <-}\StringTok{ }\KeywordTok{mean}\NormalTok{(X)}
\NormalTok{mediaX}
\end{Highlighting}
\end{Shaded}

\begin{verbatim}
## [1] 25,6
\end{verbatim}

\hypertarget{exemplo-2}{%
\subsection{Exemplo 2}\label{exemplo-2}}

Qual é a média aritmética de três provas realizadas por um aluno, cujas notas
foram 4,6 e 8. Para responder, basta aplicar \eqref{eq:mediaaritmeticasimples}.

\begin{equation*}
  \overline{X} = \frac{4+6+8}{3} = 6
\end{equation*}

\hypertarget{exemplo-2-no-r}{%
\subsection{Exemplo 2 no R}\label{exemplo-2-no-r}}

\begin{Shaded}
\begin{Highlighting}[]
\NormalTok{X2 <-}\StringTok{ }\KeywordTok{c}\NormalTok{(}\DecValTok{4}\NormalTok{, }\DecValTok{6}\NormalTok{, }\DecValTok{8}\NormalTok{)}
\NormalTok{X2}
\end{Highlighting}
\end{Shaded}

\begin{verbatim}
## [1] 4 6 8
\end{verbatim}

\begin{Shaded}
\begin{Highlighting}[]
\NormalTok{mediaX2 <-}\StringTok{ }\KeywordTok{mean}\NormalTok{(X2)}
\NormalTok{mediaX2}
\end{Highlighting}
\end{Shaded}

\begin{verbatim}
## [1] 6
\end{verbatim}

\hypertarget{muxe9dia-aritmuxe9tica-ponderada}{%
\section{Média Aritmética Ponderada}\label{muxe9dia-aritmuxe9tica-ponderada}}

Na média aritmética ponderada, cada valor pode ter importância diferentes do outros valores considerados no computo. A frequência dos valores é muito comumente usada para para dar maior ou menor importância relativa entre os valores considerados no computo da média aritmética ponderada. Veja como fica a fórmula para o cálculo da média aritmética ponderada em \eqref{eq:mediaartimeticaponderada}

\begin{equation}
    \overline{X} = \frac{1}{\sum_{i=1}^{n}w_i} \sum_{i=1}^{n} w_i X_i
    \label{eq:mediaartimeticaponderada}
\end{equation}
onde \(w_i\) é a ponderação ou peso associado a iésimo valor de \(X\).

Podemos escrever na forma de frequência relativa dos valores da variável \(X\):

\begin{equation}
  f_i = \frac{w_i}{\sum_{i=1}^{n}w_i}
  \label{eq:eq13}
\end{equation}

\hypertarget{exemplo-3}{%
\subsection{Exemplo 3}\label{exemplo-3}}

Qual é a média aritmética de um grupo de vinte alunos, oito com 22 anos, sete
de 23 anos, três de 25 anos, um de 28 anos e um de 30 anos. Para responder,
basta aplicar \eqref{eq:mediaartimeticaponderada}.

\begin{equation*}
  \overline{X} = \frac{22\times 8 + 23\times 7 + 25 \times 3 + 28 \times 1 + 
  30 \times 1}{20} = 23,5
\end{equation*}

\hypertarget{exemplo-3-no-r}{%
\subsection{Exemplo 3 no R}\label{exemplo-3-no-r}}

\begin{Shaded}
\begin{Highlighting}[]
\NormalTok{X3 <-}\StringTok{ }\KeywordTok{c}\NormalTok{(}\DecValTok{22}\NormalTok{, }\DecValTok{23}\NormalTok{, }\DecValTok{25}\NormalTok{, }\DecValTok{28}\NormalTok{, }\DecValTok{30}\NormalTok{)}
\NormalTok{X3}
\end{Highlighting}
\end{Shaded}

\begin{verbatim}
## [1] 22 23 25 28 30
\end{verbatim}

\begin{Shaded}
\begin{Highlighting}[]
\NormalTok{w3 <-}\StringTok{ }\KeywordTok{c}\NormalTok{(}\DecValTok{8}\NormalTok{, }\DecValTok{7}\NormalTok{, }\DecValTok{3}\NormalTok{, }\DecValTok{1}\NormalTok{, }\DecValTok{1}\NormalTok{)}
\NormalTok{w3}
\end{Highlighting}
\end{Shaded}

\begin{verbatim}
## [1] 8 7 3 1 1
\end{verbatim}

\begin{Shaded}
\begin{Highlighting}[]
\NormalTok{wX3 <-}\StringTok{ }\NormalTok{w3 }\OperatorTok{*}\StringTok{ }\NormalTok{X3}
\NormalTok{mediaX3 <-}\StringTok{ }\KeywordTok{sum}\NormalTok{(wX3)}\OperatorTok{/}\KeywordTok{sum}\NormalTok{(w3)}
\NormalTok{mediaX3}
\end{Highlighting}
\end{Shaded}

\begin{verbatim}
## [1] 23,5
\end{verbatim}

\hypertarget{exemplo-4}{%
\subsection{Exemplo 4}\label{exemplo-4}}

Qual é a média ponderada de três provas realizadas por um aluno, cujas notas foram 4, 6 e 8. A primeira prova tem peso igual a 1, a segunda tem peso igual a 2 e a terceira tem peso igual a 3. Para responder, basta aplicar \eqref{eq:mediaartimeticaponderada}.

\begin{equation*}
  \overline{X} = \frac{4 \times 1 + 6 \times 2 + 8 \times 3}{1 + 2 + 3} \cong 6,7
\end{equation*}

\hypertarget{exemplo-4-no-r}{%
\subsection{Exemplo 4 no R}\label{exemplo-4-no-r}}

\begin{Shaded}
\begin{Highlighting}[]
\NormalTok{X4 <-}\StringTok{ }\KeywordTok{c}\NormalTok{(}\DecValTok{4}\NormalTok{, }\DecValTok{6}\NormalTok{, }\DecValTok{8}\NormalTok{)}
\NormalTok{X4}
\end{Highlighting}
\end{Shaded}

\begin{verbatim}
## [1] 4 6 8
\end{verbatim}

\begin{Shaded}
\begin{Highlighting}[]
\NormalTok{w4 <-}\StringTok{ }\KeywordTok{c}\NormalTok{(}\DecValTok{1}\NormalTok{, }\DecValTok{2}\NormalTok{, }\DecValTok{3}\NormalTok{)}
\NormalTok{w4}
\end{Highlighting}
\end{Shaded}

\begin{verbatim}
## [1] 1 2 3
\end{verbatim}

\begin{Shaded}
\begin{Highlighting}[]
\NormalTok{wX4 <-}\StringTok{ }\NormalTok{w4 }\OperatorTok{*}\StringTok{ }\NormalTok{X4}
\NormalTok{mediaX4 <-}\StringTok{ }\KeywordTok{sum}\NormalTok{(wX4)}\OperatorTok{/}\KeywordTok{sum}\NormalTok{(w4)}
\KeywordTok{round}\NormalTok{(mediaX4, }\DataTypeTok{digits =} \DecValTok{1}\NormalTok{)}
\end{Highlighting}
\end{Shaded}

\begin{verbatim}
## [1] 6,7
\end{verbatim}

\hypertarget{muxe9dia-geomuxe9trica-simples}{%
\section{Média Geométrica Simples}\label{muxe9dia-geomuxe9trica-simples}}

Na média geométrica simples, a forma de obter uma medida resumo ou de tendência central é multiplicar todos os \(n\) valores e tirar a raiz enésima do resultado do produtório. Assim é possível ter duas fórmulas para a média geométrica a \eqref{eq:eq14} e \eqref{eq:eq15}.

\begin{equation}
    G = \left(\prod_{i=1}^{n} X_i \right)^{\frac{1}{n}}
    \label{eq:eq14}
\end{equation}
ou
\begin{equation}
    G = \sqrt[n]{X_1 \times X_2 \times \ldots \times X_n}
    \label{eq:eq15}
\end{equation}

O que acontece se um dos valores de \(X\) for igual a zero? E se um dos valores for negativo?

\hypertarget{exemplo-5-no-r}{%
\subsection{EXemplo 5 no R}\label{exemplo-5-no-r}}

\begin{Shaded}
\begin{Highlighting}[]
\NormalTok{X5 <-}\StringTok{ }\KeywordTok{c}\NormalTok{(}\DecValTok{4}\NormalTok{, }\DecValTok{6}\NormalTok{, }\DecValTok{8}\NormalTok{)}
\NormalTok{X5}
\end{Highlighting}
\end{Shaded}

\begin{verbatim}
## [1] 4 6 8
\end{verbatim}

\begin{Shaded}
\begin{Highlighting}[]
\NormalTok{n <-}\StringTok{ }\KeywordTok{length}\NormalTok{(X5)}
\NormalTok{mediaX5 <-}\StringTok{ }\KeywordTok{prod}\NormalTok{(X5)}\OperatorTok{^}\NormalTok{(}\DecValTok{1}\OperatorTok{/}\NormalTok{n)}
\KeywordTok{round}\NormalTok{(mediaX5, }\DataTypeTok{digits =} \DecValTok{1}\NormalTok{)}
\end{Highlighting}
\end{Shaded}

\begin{verbatim}
## [1] 5,8
\end{verbatim}

\hypertarget{muxe9dia-geomuxe9trica-ponderada}{%
\section{Média Geométrica Ponderada}\label{muxe9dia-geomuxe9trica-ponderada}}

Na média geométrica ponderada que podem ser calculadas através de duas fórmulas \eqref{eq:eq16} e \eqref{eq:eq17}, cada valor pode ter uma importância diferente em relação aos outros valores no computo da média geométrica. Muito comumente, esta maior ou menor importância pode estar associada a frequência dos valores considerados no cálculo.

\begin{equation}
    G = \left(\prod_{j=1}^{k} X_j^{w_j} \right)^{\frac{1}{n}}
    \label{eq:eq16}
\end{equation}
ou
\begin{equation}
    G = \sqrt[n]{X_1^{w_1} \times X_2^{w_2} \times \ldots \times X_k^{w_k}}
    \label{eq:eq17}
\end{equation}

onde a \(\sum_{j=1}^{k} w_j = n\)

\hypertarget{exemplo-6}{%
\subsection{Exemplo 6}\label{exemplo-6}}

tomando os valores do exemplo 5 e ponderando por 1,2 e 3, temos:

\begin{equation*}
  \sqrt[6]{4^1 \times 6^2 \times 8^3} \cong 6,5
\end{equation*}

\hypertarget{o-exemplo-6-no-r}{%
\subsection{O exemplo 6 no R}\label{o-exemplo-6-no-r}}

\begin{Shaded}
\begin{Highlighting}[]
\NormalTok{x6 <-}\StringTok{ }\KeywordTok{c}\NormalTok{(}\DecValTok{4}\NormalTok{, }\DecValTok{6}\NormalTok{, }\DecValTok{8}\NormalTok{)}
\KeywordTok{class}\NormalTok{(x6)}
\end{Highlighting}
\end{Shaded}

\begin{verbatim}
## [1] "numeric"
\end{verbatim}

\begin{Shaded}
\begin{Highlighting}[]
\NormalTok{x6}
\end{Highlighting}
\end{Shaded}

\begin{verbatim}
## [1] 4 6 8
\end{verbatim}

\begin{Shaded}
\begin{Highlighting}[]
\NormalTok{w6 <-}\StringTok{ }\KeywordTok{c}\NormalTok{(}\DecValTok{1}\NormalTok{, }\DecValTok{2}\NormalTok{, }\DecValTok{3}\NormalTok{)}
\NormalTok{w6}
\end{Highlighting}
\end{Shaded}

\begin{verbatim}
## [1] 1 2 3
\end{verbatim}

\begin{Shaded}
\begin{Highlighting}[]
\NormalTok{G2 <-}\StringTok{ }\KeywordTok{round}\NormalTok{((}\KeywordTok{prod}\NormalTok{(x6}\OperatorTok{^}\NormalTok{w6))}\OperatorTok{^}\NormalTok{(}\DecValTok{1}\OperatorTok{/}\KeywordTok{sum}\NormalTok{(w6)), }\DecValTok{1}\NormalTok{)}
\NormalTok{G2}
\end{Highlighting}
\end{Shaded}

\begin{verbatim}
## [1] 6,5
\end{verbatim}

\hypertarget{muxe9dia-harmuxf4nica}{%
\section{Média Harmônica}\label{muxe9dia-harmuxf4nica}}

É o inverso da média dos inversos dos valores da variável que pode ser calculada através das fórmulas \eqref{eq:mediaharmonicasimples} e \eqref{eq:mediaharmonicasimples2}.

\begin{equation}
    H = \frac{n}{\sum_{i=1}^{n}\frac{1}{X_i}}
    \label{eq:mediaharmonicasimples}
\end{equation}

\begin{equation}
    H = \frac{n}{\frac{1}{X_1} + \frac{1}{X_2} + \ldots + \frac{1}{X_n}}
    \label{eq:mediaharmonicasimples2}
\end{equation}

O que acontece se um dos valores de \(X\) for igual a zero? Para entender essa situação, use o conceito de limite fazendo o valor tender a zero.

\hypertarget{exemplo-7}{%
\subsection{Exemplo 7}\label{exemplo-7}}

Tomando o exemplo das notas, temos:

\begin{equation*}
    H = \frac{3}{\frac{1}{4} + \frac{1}{6} + \frac{1}{8}}\cong 5,5.
\end{equation*}

\hypertarget{muxe9dia-harmuxf4nica-ponderada}{%
\section{Média Harmônica Ponderada}\label{muxe9dia-harmuxf4nica-ponderada}}

Na média harmônica ponderada, assim como na média aritmética ponderada e na média geométrica ponderada, cada valor pode ter uma importância em relação aos outros valores considerados no seu cálculo. Comumente, a frequência do valor pode associaar uma maior ou menor importância no cálculo da média harmônica ponderada que pode ser calculada através das fórmulas \eqref{eq:mediaharmonicaponderada} e \eqref{eq:mediaharmonicaponderada2}

\begin{equation}
    H = \frac{n}{\sum_{j=1}^{k}w_{j}\frac{1}{X_j}}
    \label{eq:mediaharmonicaponderada}
\end{equation}
ou
\begin{equation}
    H = \frac{n}{w_1\frac{1}{X_1} + w_2\frac{1}{X_2} + \ldots + w_k \frac{1}{X_k}}
    \label{eq:mediaharmonicaponderada2}
\end{equation}

onde a \(\sum_{j=1}^{k} w_j = n\)

\hypertarget{exemplo-8}{%
\subsection{Exemplo 8}\label{exemplo-8}}

Tomando o exemplo das notas

\begin{equation*}
    H = \frac{6}{\frac{1}{4}\times 1 + \frac{1}{6}\times 2 + \frac{1}{8}\times 3}\cong 6,3.
\end{equation*}

\hypertarget{observauxe7uxe3o}{%
\subsection{Observação}\label{observauxe7uxe3o}}

Tanto para as médias simples como para as ponderadas, a
média aritmética é maior do que a média geométrica e essa, por sua vez, é maior
que a harmônica. Isso só não vale quando todos os valores são iguais. Veja de forma esquemática em \eqref{eq:diferencaentreasmedias}

\begin{equation}
  \overline{X} \geq G \geq H
  \label{eq:diferencaentreasmedias}
\end{equation}

\hypertarget{exemplo-9}{%
\subsection{Exemplo 9}\label{exemplo-9}}

O aluno tira as seguintes notas bimestrais: 3,4,5,7 e 8,5. Determine qual seria sua média final se esta fosse calculada dos três modos, aritmética, geométirca e harmônica, em cada um dos seguintes casos: i) as notas têm o mesmo peso e; ii) as notas têm pesos diferentes.

\begin{enumerate}
\def\labelenumi{\roman{enumi})}
\tightlist
\item
  As notas dos bimestres têm os mesmos pesos.
\end{enumerate}

\begin{equation*}
    \overline{X} = \frac{3 + 4,5 + 7 + 8,5}{4} = 23/4 = 5,75
  \end{equation*}
\begin{equation*}
    G = \sqrt[4]{3 \times 4,5 \times 7 \times 8,5} = \sqrt[4]{803,25} \cong 5,32
  \end{equation*}
\begin{equation*}
    H = \frac{4}{\frac{1}{3} +\frac{1}{4,5} +\frac{1}{7} +\frac{1}{8,5}}   \cong 4,90
  \end{equation*}

\begin{enumerate}
\def\labelenumi{\roman{enumi})}
\setcounter{enumi}{1}
\tightlist
\item
  Suponha que agora os pesos para as notas bimestrais sejam, 30\%, 25\%, 25\% e 20\%.
\end{enumerate}

\begin{equation*}
    \overline{X} = 0,3\times 3 + 0,25\times 4,5 + 0,25\times 7 + 0,20 \times 8,5 = 5,475
  \end{equation*}
\begin{equation*}
    G = 3^{0,3} \times 4,5^{0,25} \times 7^{0,25} \times 8,5^{0,2} = \cong 5,05
  \end{equation*}
\begin{equation*}
    H = \frac{1}{0,3\frac{1}{3} +0,25\frac{1}{4,5} +0,25\frac{1}{7} +0,2\frac{1}{8,5}}   \cong 4,66
  \end{equation*}

\hypertarget{mediana}{%
\section{Mediana}\label{mediana}}

é o valor que divide um conjunto e dados ordenados ao meio, ou seja, dois
grupos de valores de igual tamanho. Com base na definição de mediana, o valor da mediana pode ser obtida através da sua posição que proporciona duas situações: i) o número de valores é impar e ii) o número de valores é par.

\begin{enumerate}
\def\labelenumi{\roman{enumi})}
\tightlist
\item
  Quando o número de valores é impar, a posiçãodo valor correspondente a mediana é obtida através de \eqref{eq:posicaomedianaimpar}:
\end{enumerate}

\begin{equation}
  PMediana_{impar} = \dfrac{n + 1}{2}
  \label{eq:posicaomedianaimpar}
\end{equation}
onde \(n\) é o número de valores considerado no cálculo.

\begin{enumerate}
\def\labelenumi{\roman{enumi})}
\setcounter{enumi}{1}
\tightlist
\item
  Quando o número de valores é par, a posição da mediana é obtida através da média entre os dois valores centrais do conjunto de valores ordenados de menor a maior. O primeiro valor central é definido pela posição obtida através de \eqref{eq:valor1mediaposicaomedianapar}
\end{enumerate}

\begin{equation}
  P1Mediana_{par} = \dfrac{n}{2}
  \label{eq:valor1mediaposicaomedianapar}
\end{equation}
onde \(n\) é o número de valores considerado para o cálculo.

O segundo valor central é definido pelas posição obtida através de
\eqref{eq:valor2mediaposicaomedianapar}

\begin{equation}
  P2Mediana_{par} = \dfrac{n}{2} + 1
  \label{eq:valor2mediaposicaomedianapar}
\end{equation}
onde \(n\) é o número de valores considerado para o cálculo.

Assim, a mediana quando o número de valores é par é obtida através da média aritmética simples dos valores correspondentes as posições obtidas por \eqref{eq:valor1mediaposicaomedianapar} e por
\eqref{eq:valor2mediaposicaomedianapar} através de \eqref{eq:medianapar}

\begin{equation}
  Mediana_{par} = \dfrac{ValorCentral_1 + ValorCentral_2}{2}
  \label{eq:medianapar}
\end{equation}

\hypertarget{exemplo-numuxe9rico-de-mediana-quando-o-nuxfamero-de-valores-uxe9-impar}{%
\subsection{Exemplo numérico de Mediana quando o número de valores é impar}\label{exemplo-numuxe9rico-de-mediana-quando-o-nuxfamero-de-valores-uxe9-impar}}

Seja um conjunto de valores 2,-3,1,-2,0,-1,3. Obtenha a mediana.

Primeiramente ordena-se do menor para o maior.

-3,-2,-1,0,1,2,3

Como se trata de número impar de valores o valor central que divide o conjunto de valores em dois subconjuntos de igual tamanho é o valor da mediana. Neste caso é o zero.

\hypertarget{mediana-no-r}{%
\subsection{Mediana no R}\label{mediana-no-r}}

\begin{Shaded}
\begin{Highlighting}[]
\NormalTok{w <-}\StringTok{ }\KeywordTok{c}\NormalTok{(}\OperatorTok{-}\DecValTok{3}\NormalTok{, }\DecValTok{-2}\NormalTok{, }\DecValTok{-1}\NormalTok{, }\DecValTok{0}\NormalTok{, }\DecValTok{1}\NormalTok{, }\DecValTok{2}\NormalTok{, }\DecValTok{3}\NormalTok{)}
\NormalTok{mediana1 <-}\StringTok{ }\KeywordTok{median}\NormalTok{(w)}
\KeywordTok{print}\NormalTok{(mediana1)}
\end{Highlighting}
\end{Shaded}

\begin{verbatim}
## [1] 0
\end{verbatim}

\hypertarget{exemplo-numuxe9rico-de-mediana-quando-o-nuxfamero-de-valores-uxe9-par}{%
\subsection{Exemplo numérico de Mediana quando o número de valores é par}\label{exemplo-numuxe9rico-de-mediana-quando-o-nuxfamero-de-valores-uxe9-par}}

No exemplo anterior o conjunto de dados era composto por um número ímpar de valores. Neste exemplo o número de valores ordenado de menor a maior é par. Nesse caso, apesar de existir vários critérios, o mais usual é tirar a média aritmética simples entre os dois valores centrais do conjunto de valores ordenados de menor a maior. Uma vez que não existe um valor que separe dois subconjuntos de igual tamanho, a média aritmética simples destes dois valores é o valor da mediana quando o número total de valores não é impar.

Sejam os valores -2,1,3,2,-3,1. Obtenha a mediana.

Primeiramente ordena-se os seis valores.

-3,-2,-1,1,2,3

Note que trata-se de conjunto com um número par de valores.

Dessa forma, toma-se os dois valores centrais que são -1 e 1 e calcula-se a média aritmética simples. Ou seja, a mediana para este conjunto com seis valores é igual a zero.

\hypertarget{o-exemplo-do-nuxfamero-par-de-valores-no-r}{%
\subsection{O exemplo do número par de valores no R}\label{o-exemplo-do-nuxfamero-par-de-valores-no-r}}

\begin{Shaded}
\begin{Highlighting}[]
\NormalTok{v <-}\StringTok{ }\KeywordTok{c}\NormalTok{(}\OperatorTok{-}\DecValTok{3}\NormalTok{, }\DecValTok{-2}\NormalTok{, }\DecValTok{-1}\NormalTok{, }\DecValTok{1}\NormalTok{, }\DecValTok{2}\NormalTok{, }\DecValTok{3}\NormalTok{)}
\NormalTok{mediana2 <-}\StringTok{ }\KeywordTok{median}\NormalTok{(v)}
\KeywordTok{print}\NormalTok{(mediana2)}
\end{Highlighting}
\end{Shaded}

\begin{verbatim}
## [1] 0
\end{verbatim}

\hypertarget{quartis-ou-quartiles}{%
\section{Quartis ou Quartiles}\label{quartis-ou-quartiles}}

são os valores que dividem o conjunto de dados ordenados em quatro subjconjuntos
de igual tamanho. Ou seja são valores do conjunto que definem o primeiro quarto
dos dados (25\%), a metade dos dados (50\%) que coincide com a mediana, os três
quartos dos dados (75\%).

Dessa forma para obter os valores que dividem o conjunto de dados ordenados de menor a maior e quatro subconjuntos de igual tamanho, é necessário definir qual é a posição desses valores. Uma vez definido as suas posições pode-se obter os valores corretamente.

A posição do valor que separa o primeiro do segundo quartil é definido por \eqref{eq:posicaoquartilum}.

\begin{equation}
  PQ_1 = \dfrac{(n +1)}{4}
  \label{eq:posicaoquartilum}
\end{equation}
onde \(n\) é o número de valores.
A posição do valor que separa o segundo do terceiro quartil é definido por \eqref{eq:posicaoquartiltres}.

\begin{equation}
  PQ_3 = \dfrac{3(n +1)}{4}
  \label{eq:posicaoquartiltres}
\end{equation}
onde \(n\) é o número de valores.

Note que o termo genérico é percentil. Por exemplo, o quintis são os valores que dividem o conjunto de ados ordenados de menor a maior em cinco subconjuntos de igual tamanho.

\hypertarget{quartis-no-r}{%
\subsection{Quartis no R}\label{quartis-no-r}}

No R tem uma função específica para a obtenção dos quartis.

\begin{Shaded}
\begin{Highlighting}[]
\NormalTok{p <-}\StringTok{ }\KeywordTok{c}\NormalTok{(}\DecValTok{0}\OperatorTok{:}\DecValTok{100}\NormalTok{)}
\KeywordTok{length}\NormalTok{(p)}
\end{Highlighting}
\end{Shaded}

\begin{verbatim}
## [1] 101
\end{verbatim}

\begin{Shaded}
\begin{Highlighting}[]
\KeywordTok{quantile}\NormalTok{(p)}
\end{Highlighting}
\end{Shaded}

\begin{verbatim}
##   0%  25%  50%  75% 100% 
##    0   25   50   75  100
\end{verbatim}

\begin{Shaded}
\begin{Highlighting}[]
\NormalTok{faixainterquant <-}\StringTok{ }\KeywordTok{quantile}\NormalTok{(p, }\FloatTok{0.75}\NormalTok{) }\OperatorTok{-}\StringTok{ }\KeywordTok{quantile}\NormalTok{(p, }
    \FloatTok{0.25}\NormalTok{)}
\NormalTok{faixainterquant}
\end{Highlighting}
\end{Shaded}

\begin{verbatim}
## 75% 
##  50
\end{verbatim}

\hypertarget{quartis-no-r-1}{%
\subsection{Quartis no R}\label{quartis-no-r-1}}

No R tem uma função específica para a obtenção dos quartis.

\begin{Shaded}
\begin{Highlighting}[]
\NormalTok{p2 <-}\StringTok{ }\KeywordTok{c}\NormalTok{(}\DecValTok{1}\OperatorTok{:}\DecValTok{100}\NormalTok{)}
\KeywordTok{length}\NormalTok{(p2)}
\end{Highlighting}
\end{Shaded}

\begin{verbatim}
## [1] 100
\end{verbatim}

\begin{Shaded}
\begin{Highlighting}[]
\KeywordTok{quantile}\NormalTok{(p2)}
\end{Highlighting}
\end{Shaded}

\begin{verbatim}
##     0%    25%    50%    75%   100% 
##   1,00  25,75  50,50  75,25 100,00
\end{verbatim}

\begin{Shaded}
\begin{Highlighting}[]
\NormalTok{faixainterquant2 <-}\StringTok{ }\KeywordTok{quantile}\NormalTok{(p2, }\FloatTok{0.75}\NormalTok{) }\OperatorTok{-}\StringTok{ }\KeywordTok{quantile}\NormalTok{(p2, }
    \FloatTok{0.25}\NormalTok{)}
\NormalTok{faixainterquant2}
\end{Highlighting}
\end{Shaded}

\begin{verbatim}
##  75% 
## 49,5
\end{verbatim}

\hypertarget{revisuxe3o-de-literatura}{%
\chapter{Revisão de Literatura}\label{revisuxe3o-de-literatura}}

Aqui o estado da arte mundo afora.

\hypertarget{metodologia}{%
\chapter{Metodologia}\label{metodologia}}

We describe our methods in this chapter.

\hypertarget{aplicauxe7uxf5es}{%
\chapter{Aplicações}\label{aplicauxe7uxf5es}}

Some \emph{significant} applications are demonstrated in this chapter.

\hypertarget{exemplo-um}{%
\section{Exemplo um}\label{exemplo-um}}

\hypertarget{examplo-dois}{%
\section{Examplo dois}\label{examplo-dois}}

\hypertarget{considerauxe7uxf5es-finais}{%
\chapter{Considerações Finais}\label{considerauxe7uxf5es-finais}}

Terminado um excelente livro digital.

  \bibliography{estatecon.bib,book.bib,packages.bib}

\end{document}
